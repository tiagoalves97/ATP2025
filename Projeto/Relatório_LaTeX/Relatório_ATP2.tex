\documentclass[12pt,a4paper]{article}

% ----------------------------
% PACOTES
% ----------------------------
\usepackage[utf8]{inputenc}
\usepackage[T1]{fontenc}
\usepackage[portuguese]{babel}
\usepackage{graphicx}
\usepackage{amsmath}
\usepackage{float}
\usepackage{booktabs}
\usepackage{geometry}
\usepackage{setspace}
\usepackage{hyperref}
\usepackage{caption}

\geometry{margin=2.5cm}
\onehalfspacing

% ----------------------------
% CAPA
% ----------------------------
\title{
    \textbf{Simulação de uma Clínica Médica} \\
    \large Projeto de Algoritmos e Técnicas de Programação
}

\author{
    Grupo 23 \\
    Inês Pereira - a111776 \\
    Tiago Alves - a111017 \\
    \vspace{0.5cm}
    Tomás Gonçalves - a111441 \\
    Licenciatura em Engenharia Biomédica \\
    Universidade do Minho
}

\date{Ano Letivo 2025/26}

\begin{document}

\maketitle
\thispagestyle{empty}
\newpage

% ----------------------------
% RESUMO
% ----------------------------
\section*{Resumo}

Este trabalho apresenta o desenvolvimento de um sistema de simulação discreta para o funcionamento de uma clínica médica, com especial foco na gestão de filas de espera, alocação de médicos e aplicação do sistema de Triagem de Manchester. O objetivo principal consiste em analisar o impacto de diferentes parâmetros operacionais, tais como a taxa de chegada de doentes, o número de médicos disponíveis e a distribuição das especialidades, no desempenho global do sistema. A simulação foi implementada em Python, recorrendo a bibliotecas como \textit{NumPy}, \textit{Matplotlib} e \textit{FreeSimpleGUI}, permitindo não só a análise estatística dos resultados, mas também a visualização gráfica e interativa do processo.

\newpage

% ----------------------------
% ÍNDICE
% ----------------------------
\tableofcontents
\newpage

% ----------------------------
% INTRODUÇÃO
% ----------------------------
\section{Introdução}

A gestão eficiente de serviços de saúde é um desafio complexo, especialmente em contextos de elevada procura, como serviços de urgência e clínicas médicas. Um dos principais problemas enfrentados nestes sistemas é a formação de filas de espera, que pode resultar em tempos de espera elevados, insatisfação dos doentes e sobrecarga dos profissionais de saúde.

Neste contexto, a simulação computacional surge como uma ferramenta poderosa para analisar e compreender o comportamento destes sistemas sem interferir com o seu funcionamento real. Este trabalho propõe a modelação e simulação de uma clínica médica, incorporando o sistema de Triagem de Manchester, amplamente utilizado em serviços de urgência para priorizar o atendimento dos doentes com base na gravidade clínica.

O projeto foi desenvolvido em grupo, promovendo a divisão de tarefas, discussão de soluções e validação conjunta dos resultados obtidos.

% ----------------------------
% OBJETIVOS
% ----------------------------
\section{Objetivos}

Os principais objetivos deste trabalho são:

\begin{itemize}
    \item Modelar o funcionamento de uma clínica médica através de simulação discreta.
    \item Implementar o sistema de Triagem de Manchester como critério de prioridade.
    \item Analisar o impacto da especialização dos médicos na gestão das filas.
    \item Avaliar métricas de desempenho como tempo médio de espera, ocupação dos médicos e número de doentes atendidos.
    \item Desenvolver uma interface gráfica interativa para visualização do sistema.
\end{itemize}

% ----------------------------
% DESCRIÇÃO DO SISTEMA
% ----------------------------
\section{Descrição do Sistema Simulado}

O sistema simulado representa uma clínica médica composta por diferentes especialidades: Medicina Geral, Pediatria e Ortopedia. Os doentes chegam ao sistema segundo uma distribuição probabilística e são encaminhados para a respetiva fila de acordo com a sua especialidade.

Após a chegada, cada doente é classificado segundo o sistema de Triagem de Manchester, sendo atribuída uma pulseira de prioridade (Vermelha, Laranja, Amarela, Verde ou Azul). Os médicos atendem os doentes respeitando sempre a prioridade clínica dentro da sua especialidade.

% ----------------------------
% PARÂMETROS DA SIMULAÇÃO
% ----------------------------
\section{Parâmetros da Simulação}

Os principais parâmetros configuráveis no sistema são:

\begin{itemize}
    \item Número de médicos disponíveis.
    \item Taxa média de chegada de doentes.
    \item Distribuição estatística das chegadas (Exponencial, Normal ou Uniforme).
    \item Tempo médio de consulta.
    \item Duração total da simulação.
\end{itemize}

Estes parâmetros podem ser alterados dinamicamente através da interface gráfica, permitindo testar diferentes cenários operacionais.

% ----------------------------
% IMPLEMENTAÇÃO
% ----------------------------
\section{Implementação}

A simulação foi implementada em Python, utilizando uma abordagem orientada a eventos discretos. O estado do sistema é atualizado a cada unidade de tempo, sendo registadas métricas relevantes para posterior análise estatística.

A interface gráfica foi desenvolvida com a biblioteca \textit{FreeSimpleGUI}, proporcionando uma visualização em tempo real das filas de espera, estado dos médicos e estatísticas globais da simulação.

% ----------------------------
% RESULTADOS
% ----------------------------
\section{Resultados}

Nesta secção apresentam-se os principais resultados obtidos através da execução da simulação.

\subsection{Evolução do Tamanho da Fila}

\begin{figure}[H]
    \centering
    \includegraphics[width=0.8\textwidth]{fila.png}
    \caption{Evolução do tamanho total da fila ao longo do tempo}
\end{figure}

A análise da evolução do tamanho da fila de espera ao longo do tempo permite compreender a dinâmica do sistema face à taxa de chegada de doentes e à capacidade de atendimento disponível. Observa-se que, em determinados períodos, a fila aumenta de forma significativa, refletindo momentos em que a chegada de novos doentes supera temporariamente a capacidade dos médicos.

Este comportamento é esperado em sistemas reais de saúde e demonstra que o modelo consegue reproduzir situações de sobrecarga. Ao longo do tempo, verifica-se também a estabilização ou redução da fila, indicando que o sistema é capaz de recuperar quando os recursos disponíveis conseguem absorver a procura acumulada.

\subsection{Ocupação dos Médicos}

\begin{figure}[H]
    \centering
    \includegraphics[width=0.8\textwidth]{ocupacao.png}
    \caption{Taxa de ocupação dos médicos}
\end{figure}

O gráfico da ocupação dos médicos ilustra a proporção de profissionais em atendimento ao longo da simulação. É possível observar períodos de elevada ocupação, próximos da capacidade máxima, alternados com momentos de menor carga de trabalho.

Este resultado demonstra uma utilização eficiente dos recursos humanos, evitando tanto a ociosidade excessiva como a saturação constante. A variação ao longo do tempo está diretamente relacionada com o padrão de chegadas dos doentes e com a duração das consultas, refletindo um comportamento realista do funcionamento de uma clínica médica.

\subsection{Distribuição por Pulseira de Triagem}

\begin{figure}[H]
    \centering
    \includegraphics[width=0.8\textwidth]{hist_pulseiras.png}
    \caption{Número de doentes atendidos por pulseira de Manchester}
\end{figure}

A distribuição dos doentes atendidos por pulseira de triagem permite avaliar o impacto da Triagem de Manchester na organização do atendimento. Observa-se que a maioria dos doentes pertence às categorias Amarela, Verde e Azul, o que está de acordo com a realidade clínica, onde situações menos urgentes são mais frequentes.

Apesar da menor frequência de casos classificados como Vermelha e Laranja, estes representam situações de maior gravidade clínica, sendo corretamente identificados e integrados no sistema. Este resultado confirma a coerência da distribuição de prioridades implementada na simulação.

\subsection{Tempo Médio de Espera por Pulseira}

Para além do número de doentes atendidos por nível de prioridade, é fundamental analisar o tempo médio de espera associado a cada pulseira de triagem. Esta métrica permite avaliar se o sistema está a respeitar o princípio base da Triagem de Manchester: os doentes mais graves devem ser atendidos com maior rapidez.

\begin{figure}[H]
    \centering
    \includegraphics[width=0.8\textwidth]{espera_pulseira.png}
    \caption{Tempo médio de espera por pulseira de triagem}
\end{figure}

A análise do gráfico permite observar uma clara relação inversa entre a prioridade clínica e o tempo médio de espera. Doentes classificados com pulseira Vermelha e Laranja apresentam tempos de espera significativamente inferiores quando comparados com os doentes das categorias Verde e Azul. Este comportamento valida a correta implementação do sistema de prioridades e demonstra que a lógica de atendimento respeita critérios clínicos realistas.

Adicionalmente, este resultado evidencia a importância da triagem em ambientes com recursos limitados, uma vez que permite otimizar o atendimento e reduzir riscos associados a atrasos em situações críticas.

\subsection{Atendimentos por Especialidade}

\begin{figure}[H]
    \centering
    \includegraphics[width=0.8\textwidth]{hist_especialidades.png}
    \caption{Doentes atendidos por especialidade}
\end{figure}

O gráfico referente aos doentes atendidos por especialidade permite analisar a distribuição da carga de trabalho entre as diferentes áreas clínicas. Verifica-se que a especialidade Geral apresenta o maior número de atendimentos, refletindo o seu papel central numa clínica médica.

As especialidades de Pediatria e Ortopedia apresentam valores inferiores, mas consistentes com a probabilidade de encaminhamento definida no modelo. Este resultado demonstra que o sistema consegue representar de forma equilibrada a diversidade de casos clínicos e a sua afetação às respetivas especialidades.

\subsection{Tempo Médio de Espera por Especialidade}

\begin{figure}[H]
    \centering
    \includegraphics[width=0.8\textwidth]{espera_especialidade.png}
    \caption{Tempo médio de espera por especialidade}
\end{figure}

A análise do tempo médio de espera por especialidade permite identificar diferenças na eficiência do atendimento entre as várias áreas clínicas. Observam-se variações nos tempos médios, que podem ser explicadas pela disponibilidade de médicos, pelo número de doentes encaminhados e pela complexidade dos casos associados a cada especialidade.

Este resultado evidencia a importância de uma correta alocação de recursos humanos por especialidade, uma vez que desequilíbrios podem conduzir a aumentos significativos no tempo de espera. A inclusão desta análise reforça a utilidade do modelo como ferramenta de apoio à tomada de decisão.


% ----------------------------
% DISCUSSÃO
% ----------------------------
\section{Discussão Crítica}

A simulação desenvolvida permitiu analisar, de forma detalhada, o funcionamento de uma clínica médica sujeita a diferentes padrões de chegada de doentes, prioridades clínicas e limitações de recursos humanos. Os resultados obtidos demonstram que o modelo é capaz de reproduzir comportamentos realistas, tais como o crescimento temporário das filas, a variação da ocupação dos médicos e a influência direta da triagem no tempo de espera dos doentes.

A análise dos gráficos evidenciou que o sistema de prioridades baseado na Triagem de Manchester desempenha um papel fundamental na organização do atendimento. Doentes com maior prioridade clínica apresentaram tempos de espera significativamente inferiores, validando a correta implementação da lógica de atendimento e demonstrando a sua importância em ambientes com elevada procura e recursos limitados.

No entanto, a simulação também revelou limitações inerentes à distribuição de médicos por especialidade. Em determinados cenários, especialidades com menor número de profissionais apresentaram tempos médios de espera superiores, mesmo para doentes com prioridade intermédia. Este comportamento evidencia a necessidade de uma alocação equilibrada de recursos humanos e demonstra como pequenas variações na disponibilidade podem ter impacto significativo na eficiência do sistema.

Outro aspeto relevante prende-se com a utilização de distribuições probabilísticas para modelar a chegada dos doentes e a duração das consultas. Embora estas abordagens permitam introduzir variabilidade e realismo, assumem simplificações que não capturam toda a complexidade de um ambiente clínico real, como interrupções inesperadas, urgências extremas ou variações sazonais da procura.

Apesar destas limitações, o modelo constitui uma ferramenta valiosa para análise exploratória e apoio à tomada de decisão, permitindo testar diferentes configurações de parâmetros e avaliar o seu impacto no desempenho global da clínica. A simulação demonstra, assim, o potencial da modelação computacional como instrumento de apoio ao planeamento e gestão de sistemas de saúde.

% ----------------------------
% TRABALHO DE GRUPO
% ----------------------------
\section{Organização do Trabalho de Grupo}

O trabalho foi desenvolvido por três elementos, com uma divisão equilibrada de tarefas:

\begin{itemize}
    \item Modelação do sistema e lógica da simulação.
    \item Desenvolvimento da interface gráfica.
    \item Análise de resultados e elaboração do relatório.
\end{itemize}

A colaboração entre os elementos foi essencial para a validação do modelo e melhoria contínua do sistema.

% ----------------------------
% CONCLUSÃO
% ----------------------------
\section{Conclusão}

Neste trabalho foi desenvolvido um modelo de simulação discreta para o funcionamento de uma clínica médica, integrando conceitos de triagem clínica, gestão de filas, alocação de recursos e análise estatística de desempenho. A implementação permitiu simular a chegada de doentes segundo diferentes distribuições probabilísticas, o atendimento por médicos especializados e a priorização baseada na Triagem de Manchester.

Os resultados obtidos evidenciam que o sistema responde de forma adequada às prioridades clínicas, garantindo tempos de espera reduzidos para doentes em situação mais grave. A análise dos gráficos confirmou a coerência do modelo, tanto na evolução das filas de espera como na ocupação dos médicos, refletindo comportamentos observados em contextos reais de prestação de cuidados de saúde.

Adicionalmente, a inclusão de métricas como o tempo médio de espera por pulseira e por especialidade permitiu uma avaliação mais profunda do desempenho do sistema, destacando potenciais pontos de melhoria na distribuição de recursos. Estes resultados demonstram a importância da simulação como ferramenta de apoio à tomada de decisão, especialmente em cenários onde alterações reais seriam dispendiosas ou difíceis de implementar.

Como trabalho futuro, seria interessante expandir o modelo para incluir fatores adicionais, tais como a distinção entre médicos sénior e júnior, a introdução de urgências extremas, variações temporais na taxa de chegada ou a simulação de múltiplos dias de funcionamento. Estas extensões permitiriam aumentar ainda mais o realismo do modelo e a sua aplicabilidade a contextos reais.

Em suma, o trabalho desenvolvido cumpre os objetivos propostos, apresentando uma solução funcional, bem estruturada e capaz de analisar de forma crítica o funcionamento de um sistema clínico complexo, contribuindo para a compreensão dos desafios associados à gestão eficiente de serviços de saúde.

\end{document}
