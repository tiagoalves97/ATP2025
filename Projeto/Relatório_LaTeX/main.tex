\documentclass[12pt,a4paper]{article}

\usepackage[portuguese]{babel}
\usepackage[utf8]{inputenc}
\usepackage[T1]{fontenc}
\usepackage{geometry}
\usepackage{graphicx}
\usepackage{float}
\usepackage{amsmath}
\usepackage{booktabs}
\usepackage{hyperref}

\geometry{margin=2.5cm}

\title{\textbf{Simulação de uma Clínica Médica}\\
\large Projeto de Algoritmos e Técnicas de Programação}

\author{
Grupo 23 \\
\vspace{0.3cm}
Inês Pereira – A111776 \\
\vspace{0.3cm}
Tiago Alves – A111017 \\
\vspace{0.3cm}
Tomás Gonçalves – A111441 \\
\vspace{0.3cm}
Licenciatura em Engenharia Biomédica \\
Universidade do Minho
}

\date{Ano Letivo 2025/26}

\begin{document}

\maketitle
\thispagestyle{empty}
\newpage

\tableofcontents
\newpage

% ==================================================
\section{Introdução}

No âmbito da unidade curricular de Algoritmos e Técnicas de Programação, desenvolvemos em grupo uma aplicação em Python que simula o funcionamento de uma clínica médica. O projeto tem como foco a gestão de filas de espera, a alocação de médicos por especialidade e a priorização clínica dos doentes.

A simulação baseia-se em eventos discretos e integra o Sistema de Triagem de Manchester, amplamente utilizado em serviços de urgência hospitalar. O objetivo principal é analisar métricas de desempenho como tempos de espera, ocupação dos médicos e impacto da prioridade clínica no atendimento.

% ==================================================
\section{Objetivos do Projeto}

Os objetivos definidos para este trabalho de grupo são:

\begin{itemize}
    \item Desenvolver uma simulação de eventos discretos em Python;
    \item Modelar a chegada e atendimento de doentes num contexto clínico;
    \item Aplicar distribuições estatísticas aos processos de chegada;
    \item Implementar a Triagem de Manchester como sistema de priorização;
    \item Recolher e analisar estatísticas de desempenho do sistema;
    \item Criar uma interface gráfica interativa para visualização e controlo.
\end{itemize}

% ==================================================
\section{Descrição do Sistema}

O sistema desenvolvido simula uma clínica médica composta por várias especialidades: Medicina Geral, Pediatria e Ortopedia. Os doentes chegam à clínica de forma aleatória e são classificados segundo a Triagem de Manchester, sendo encaminhados para a fila correspondente à especialidade necessária.

Os médicos pertencem a uma única especialidade e atendem os doentes de acordo com a prioridade clínica e disponibilidade. Caso não exista médico disponível, o doente permanece em fila de espera até ser atendido.

% ==================================================
\section{Triagem de Manchester}

A priorização dos doentes é realizada com base no Sistema de Triagem de Manchester, que classifica os doentes em cinco níveis de urgência, representados por pulseiras de diferentes cores:

\begin{table}[H]
\centering
\begin{tabular}{ll}
\toprule
\textbf{Pulseira} & \textbf{Prioridade} \\
\midrule
Vermelha & Emergência \\
Laranja & Muito urgente \\
Amarela & Urgente \\
Verde & Pouco urgente \\
Azul & Não urgente \\
\bottomrule
\end{tabular}
\caption{Classificação segundo a Triagem de Manchester}
\end{table}

Este sistema garante que os doentes em situação mais crítica são atendidos prioritariamente.

% ==================================================
\section{Parâmetros da Simulação}

A aplicação permite configurar os seguintes parâmetros através da interface gráfica:

\begin{itemize}
    \item Número de médicos disponíveis;
    \item Taxa média de chegada de doentes;
    \item Distribuição estatística das chegadas (Exponencial, Normal ou Uniforme);
    \item Tempo médio de consulta;
    \item Duração total da simulação.
\end{itemize}

Esta flexibilidade permite analisar diferentes cenários e avaliar o impacto das variáveis no desempenho do sistema.

% ==================================================
\section{Modelo de Simulação}

A simulação é baseada num modelo de eventos discretos que avança o tempo em passos unitários. Em cada passo são tratados os seguintes eventos:

\begin{enumerate}
    \item Chegada de novos doentes;
    \item Atualização do estado dos médicos;
    \item Conclusão de consultas;
    \item Atribuição de médicos livres a doentes em espera;
    \item Atualização das estatísticas globais.
\end{enumerate}

As filas de espera são separadas por especialidade, assegurando coerência entre médicos e doentes.

% ==================================================
\section{Interface Gráfica}

A interface gráfica foi desenvolvida com a biblioteca \texttt{FreeSimpleGUI}. Esta permite ao utilizador:

\begin{itemize}
    \item Configurar os parâmetros da simulação;
    \item Iniciar e parar a execução do sistema;
    \item Visualizar em tempo real as filas de espera e o estado dos médicos;
    \item Consultar estatísticas agregadas do sistema;
    \item Gerar gráficos e histogramas;
    \item Exportar os resultados para ficheiros CSV.
\end{itemize}

A organização visual da interface melhora a usabilidade e facilita a análise dos resultados.

% ==================================================
\section{Resultados e Análise}

Os testes realizados pelo grupo permitiram observar vários comportamentos relevantes:

\begin{itemize}
    \item O aumento da taxa de chegada provoca crescimento do tamanho médio das filas;
    \item O aumento do número de médicos reduz o tempo médio de espera;
    \item Doentes com maior prioridade apresentam tempos de espera significativamente inferiores;
    \item A ocupação média dos médicos aumenta com taxas de chegada mais elevadas;
    \item Desequilíbrios entre especialidades podem causar congestionamento localizado.
\end{itemize}

Foram gerados gráficos temporais e histogramas que permitiram uma análise detalhada do desempenho do sistema.

% ==================================================
\section{Distribuição de Tarefas no Grupo}

O trabalho foi desenvolvido de forma colaborativa pelos três elementos do grupo, com a seguinte distribuição de responsabilidades:

\begin{itemize}
    \item \textbf{Elemento 1}: Desenvolvimento da lógica da simulação e gestão de filas;
    \item \textbf{Elemento 2}: Implementação da Triagem de Manchester e recolha de estatísticas;
    \item \textbf{Elemento 3}: Desenvolvimento da interface gráfica e análise dos resultados.
\end{itemize}

Todos os elementos participaram na validação do sistema e na elaboração do relatório final.

% ==================================================
\section{Conclusão}

O trabalho desenvolvido permitiu aplicar conceitos fundamentais de programação, estatística e simulação computacional. A integração da Triagem de Manchester tornou o modelo mais realista e adequado ao contexto clínico.

Os objetivos definidos para o projeto foram cumpridos com sucesso, resultando numa aplicação funcional, extensível e com elevado valor pedagógico.

\end{document}
